%%%%%%%%%%%%%%%%%%%%%%%%%%%%%%%%%%%%%%%%%%%%%%%%%%%%%%%%%%%%%%%%%%%%%%%%%%%%%%%
%%%%%%%%%%%%%%%%%%%%%%%%%%%%%%%%%%%%%%%%%%%%%%%%%%%%%%%%%%%%%%%%%%%%%%%%%%%%%%%
%
%   Define.tex
%   ==========
%
%   By Yonghwan Oh
%
%   Oct. 29, 1995
%
%   Robotics. Lab.
%   School of Mechanical Engineering
%   Pohang University of Science & Technology
%
%%%%%%%%%%%%%%%%%%%%%%%%%%%%%%%%%%%%%%%%%%%%%%%%%%%%%%%%%%%%%%%%%%%%%%%%%%%%%%%
%%%%%%%%%%%%%%%%%%%%%%%%%%%%%%%%%%%%%%%%%%%%%%%%%%%%%%%%%%%%%%%%%%%%%%%%%%%%%%%
\input{epsf.sty}    %   To input figure
\sloppypar
\flushbottom
\newtheorem{proposition}{Proposition}
\newtheorem{definition}{Definition}
\newtheorem{lemma}{Lemma}
\newtheorem{theorem}{Theorem}
\newtheorem{property}{Property}
\newtheorem{assumption}{Assumption}
\newtheorem{remark}{\underline{\sf REMARKS}}[section]
%
%   New Definition
%   ==============
%
\def\notation{
    \sloppypar
    \thispagestyle{empty}
    {\underline{\Huge\bf Nomenclature}}     \\
    \vspace*{1cm}                   \\
    \tabular{rcl}
}
%\let\endnotation=\endtabular
\def\endnotation{
%   \endtabular\par\newpage \def\arraystretch{1.0}\setcounter{page}{1}
    \endtabular\par\newpage \setcounter{page}{1}%
}

\def\magscale#1{scaled \magstep #1}
\def\halfmag{scaled \magstephalf}
\def\ptscale#1{scaled #100}
%
%   Definition for boldmath type
%   ============================
%
\def\0{\mathbf{0}}
\def\1{\mathbf{1}}
\def\2{\mathbf{2}}
\def\3{\mathbf{3}}
\def\4{\mathbf{4}}
\def\5{\mathbf{5}}
\def\6{\mathbf{6}}
\def\7{\mathbf{7}}
\def\8{\mathbf{8}}
\def\9{\mathbf{9}}
%
\def\A{\mbox{\boldmath$A$}}
\def\B{\mbox{\boldmath$B$}}
\def\C{\mbox{\boldmath$C$}}
\def\D{\mbox{\boldmath$D$}}
\def\E{\mbox{\boldmath$E$}}
\def\F{\mbox{\boldmath$F$}}
\def\G{\mbox{\boldmath$G$}}
\def\H{\mbox{\boldmath$H$}}
\def\I{\mbox{\boldmath$I$}}
\def\J{\mbox{\boldmath$J$}}
\def\K{\mbox{\boldmath$K$}}
\def\L{\mbox{\boldmath$L$}}
\def\M{\mbox{\boldmath$M$}}
\def\N{\mbox{\boldmath$N$}}
\def\O{\mbox{\boldmath$O$}}
\def\P{\mbox{\boldmath$P$}}
\def\Q{\mbox{\boldmath$Q$}}
\def\R{\mbox{\boldmath$R$}}
\def\S{\mbox{\boldmath$S$}}
\def\T{\mbox{\boldmath$T$}}
\def\U{\mbox{\boldmath$U$}}
\def\V{\mbox{\boldmath$V$}}
\def\W{\mbox{\boldmath$W$}}
\def\X{\mbox{\boldmath$X$}}
\def\Y{\mbox{\boldmath$Y$}}
\def\Z{\mbox{\boldmath$Z$}}
%
\def\a{\mbox{\boldmath$a$}}
\def\b{\mbox{\boldmath$b$}}
\def\c{\mbox{\boldmath$c$}}
\def\d{\mbox{\boldmath$d$}}
\def\e{\mbox{\boldmath$e$}}
\def\f{\mbox{\boldmath$f$}}
\def\g{\mbox{\boldmath$g$}}
\def\h{\mbox{\boldmath$h$}}
\def\i{\mbox{\boldmath$i$}}
\def\j{\mbox{\boldmath$j$}}
\def\k{\mbox{\boldmath$k$}}
\def\l{\mbox{\boldmath$l$}}
\def\m{\mbox{\boldmath$m$}}
\def\n{\mbox{\boldmath$n$}}
\def\o{\mbox{\boldmath$o$}}
\def\p{\mbox{\boldmath$p$}}
\def\q{\mbox{\boldmath$q$}}
\def\r{\mbox{\boldmath$r$}}
\def\s{\mbox{\boldmath$s$}}
\def\t{\mbox{\boldmath$t$}}
\def\u{\mbox{\boldmath$u$}}
\def\v{\mbox{\boldmath$v$}}
\def\w{\mbox{\boldmath$w$}}
\def\x{\mbox{\boldmath$x$}}
\def\y{\mbox{\boldmath$y$}}
\def\z{\mbox{\boldmath$z$}}
%
\def\bDel{\mbox{\boldmath$\Delta$}}
\def\bGamma{\mbox{\boldmath$\Gamma$}}
\def\bLambda{\mbox{\large\boldmath$\Lambda$}}
\def\bOmega{\mbox{\boldmath$\Omega$}}
\def\bPhi{\mbox{\boldmath$\Phi$}}
\def\bPi{\mbox{\boldmath$\Pi$}}
\def\bPsi{\mbox{\boldmath$\Psi$}}
\def\bSigma{\mbox{\boldmath$\Sigma$}}
\def\bTheta{\mbox{\boldmath$\Theta$}}
\def\bUpsilon{\mbox{\boldmath$\Upsilon$}}
\def\bXi{\mbox{\boldmath$\Xi$}}
%
\def\balpha{\mbox{\boldmath$\alpha$}}
\def\bbeta{\mbox{\boldmath$\beta$}}
\def\bdel{\mbox{\boldmath$\delta$}}
\def\bgamma{\mbox{\boldmath$\gamma$}}
\def\bepsilon{\mbox{\boldmath$\epsilon$}}
\def\bvarepsilon{\mbox{\boldmath$\varepsilon$}}
\def\bzeta{\mbox{\boldmath$\zeta$}}
\def\Beta{\mbox{\boldmath$\eta$}}
\def\btheta{\mbox{\boldmath$\theta$}}
\def\bvartheta{\mbox{\boldmath$\vartheta$}}
\def\biota{\mbox{\boldmath$\iota$}}
\def\bkappa{\mbox{\boldmath$\kappa$}}
\def\blambda{\mbox{\boldmath$\lambda$}}
\def\bmu{\mbox{\boldmath$\mu$}}
\def\bnu{\mbox{\boldmath$\nu$}}
\def\bnabla{\mbox{\boldmath$\nabla$}}
\def\bxi{\mbox{\boldmath$\xi$}}
\def\bpi{\mbox{\boldmath$\pi$}}
\def\bvarpi{\mbox{\boldmath$\varpi$}}
\def\brho{\mbox{\boldmath$\rho$}}
\def\bvarrho{\mbox{\boldmath$\varrho$}}
\def\bsigma{\mbox{\boldmath$\sigma$}}
\def\bvarsigma{\mbox{\boldmath$\varsigma$}}
\def\btau{\mbox{\boldmath$\tau$}}
\def\bupsilon{\mbox{\boldmath$\upsilon$}}
\def\bphi{\mbox{\boldmath$\phi$}}
\def\bvarphi{\mbox{\boldmath$\varphi$}}
\def\bchi{\mbox{\boldmath$\chi$}}
\def\bpsi{\mbox{\boldmath$\psi$}}
\def\bomega{\mbox{\boldmath$\omega$}}
%
\def\tA{\mbox{\boldmath$\tilde{A}$}}
\def\tB{\mbox{\boldmath$\tilde{B}$}}
\def\tC{\mbox{\boldmath$\tilde{C}$}}
\def\tD{\mbox{\boldmath$\tilde{D}$}}
\def\tE{\mbox{\boldmath$\tilde{E}$}}
\def\tF{\mbox{\boldmath$\tilde{F}$}}
\def\tG{\mbox{\boldmath$\tilde{G}$}}
\def\tH{\mbox{\boldmath$\tilde{H}$}}
\def\tI{\mbox{\boldmath$\tilde{I}$}}
\def\tJ{\mbox{\boldmath$\tilde{J}$}}
\def\tK{\mbox{\boldmath$\tilde{K}$}}
\def\tL{\mbox{\boldmath$\tilde{L}$}}
\def\tM{\mbox{\boldmath$\tilde{M}$}}
\def\tN{\mbox{\boldmath$\tilde{N}$}}
\def\tO{\mbox{\boldmath$\tilde{O}$}}
\def\tP{\mbox{\boldmath$\tilde{P}$}}
\def\tQ{\mbox{\boldmath$\tilde{Q}$}}
\def\tR{\mbox{\boldmath$\tilde{R}$}}
\def\tS{\mbox{\boldmath$\tilde{S}$}}
\def\tT{\mbox{\boldmath$\tilde{T}$}}
\def\tU{\mbox{\boldmath$\tilde{U}$}}
\def\tV{\mbox{\boldmath$\tilde{V}$}}
\def\tW{\mbox{\boldmath$\tilde{W}$}}
\def\tX{\mbox{\boldmath$\tilde{X}$}}
\def\tY{\mbox{\boldmath$\tilde{Y}$}}
\def\tZ{\mbox{\boldmath$\tilde{Z}$}}
%
\def\ta{\mbox{\boldmath$\tilde{a}$}}
\def\tb{\mbox{\boldmath$\tilde{b}$}}
\def\tc{\mbox{\boldmath$\tilde{c}$}}
\def\td{\mbox{\boldmath$\tilde{d}$}}
\def\te{\mbox{\boldmath$\tilde{e}$}}
\def\tf{\mbox{\boldmath$\tilde{f}$}}
\def\tg{\mbox{\boldmath$\tilde{g}$}}
\def\th{\mbox{\boldmath$\tilde{h}$}}
\def\ti{\mbox{\boldmath$\tilde{i}$}}
\def\tj{\mbox{\boldmath$\tilde{j}$}}
\def\tk{\mbox{\boldmath$\tilde{k}$}}
\def\tl{\mbox{\boldmath$\tilde{l}$}}
\def\tm{\mbox{\boldmath$\tilde{m}$}}
\def\tn{\mbox{\boldmath$\tilde{n}$}}
\def\to{\mbox{\boldmath$\tilde{o}$}}
\def\tp{\mbox{\boldmath$\tilde{p}$}}
\def\tq{\mbox{\boldmath$\tilde{q}$}}
\def\tr{\mbox{\boldmath$\tilde{r}$}}
\def\ts{\mbox{\boldmath$\tilde{s}$}}
\def\tu{\mbox{\boldmath$\tilde{u}$}}
\def\tv{\mbox{\boldmath$\tilde{v}$}}
\def\tw{\mbox{\boldmath$\tilde{w}$}}
\def\tx{\mbox{\boldmath$\tilde{x}$}}
\def\ty{\mbox{\boldmath$\tilde{y}$}}
\def\tz{\mbox{\boldmath$\tilde{z}$}}
%
\def\tBeta{\mbox{\boldmath$\tilde{\eta}$}}
\def\tbLambda{\mbox{\boldmath$\tilde{\Lambda}$}}
%
\def\Ah{\mbox{\boldmath$\hat{A}$}}
\def\Bh{\mbox{\boldmath$\hat{B}$}}
\def\Ch{\mbox{\boldmath$\hat{C}$}}
\def\Dh{\mbox{\boldmath$\hat{D}$}}
\def\Eh{\mbox{\boldmath$\hat{E}$}}
\def\Fh{\mbox{\boldmath$\hat{F}$}}
\def\Gh{\mbox{\boldmath$\hat{G}$}}
\def\Hh{\mbox{\boldmath$\hat{H}$}}
\def\Ih{\mbox{\boldmath$\hat{I}$}}
\def\Jh{\mbox{\boldmath$\hat{J}$}}
\def\Kh{\mbox{\boldmath$\hat{K}$}}
\def\Lh{\mbox{\boldmath$\hat{L}$}}
\def\Mh{\mbox{\boldmath$\hat{M}$}}
\def\Nh{\mbox{\boldmath$\hat{N}$}}
\def\Oh{\mbox{\boldmath$\hat{O}$}}
\def\Ph{\mbox{\boldmath$\hat{P}$}}
\def\Qh{\mbox{\boldmath$\hat{Q}$}}
\def\Rh{\mbox{\boldmath$\hat{R}$}}
\def\Sh{\mbox{\boldmath$\hat{S}$}}
\def\Th{\mbox{\boldmath$\hat{T}$}}
\def\Uh{\mbox{\boldmath$\hat{U}$}}
\def\Vh{\mbox{\boldmath$\hat{V}$}}
\def\Wh{\mbox{\boldmath$\hat{W}$}}
\def\Xh{\mbox{\boldmath$\hat{X}$}}
\def\Yh{\mbox{\boldmath$\hat{Y}$}}
\def\Zh{\mbox{\boldmath$\hat{Z}$}}
%
\def\ah{\mbox{\boldmath$\hat{a}$}}
\def\bh{\mbox{\boldmath$\hat{b}$}}
\def\ch{\mbox{\boldmath$\hat{c}$}}
\def\dh{\mbox{\boldmath$\hat{d}$}}
\def\eh{\mbox{\boldmath$\hat{e}$}}
\def\fh{\mbox{\boldmath$\hat{f}$}}
\def\gh{\mbox{\boldmath$\hat{g}$}}
\def\hh{\mbox{\boldmath$\hat{h}$}}
\def\ih{\mbox{\boldmath$\hat{i}$}}
\def\jh{\mbox{\boldmath$\hat{j}$}}
\def\kh{\mbox{\boldmath$\hat{k}$}}
\def\lh{\mbox{\boldmath$\hat{l}$}}
\def\mh{\mbox{\boldmath$\hat{m}$}}
\def\nh{\mbox{\boldmath$\hat{n}$}}
\def\oh{\mbox{\boldmath$\hat{o}$}}
\def\ph{\mbox{\boldmath$\hat{p}$}}
\def\qh{\mbox{\boldmath$\hat{q}$}}
\def\rh{\mbox{\boldmath$\hat{r}$}}
\def\sh{\mbox{\boldmath$\hat{s}$}}
\def\th{\mbox{\boldmath$\hat{t}$}}
\def\uh{\mbox{\boldmath$\hat{u}$}}
\def\vh{\mbox{\boldmath$\hat{v}$}}
\def\wh{\mbox{\boldmath$\hat{w}$}}
\def\xh{\mbox{\boldmath$\hat{x}$}}
\def\yh{\mbox{\boldmath$\hat{y}$}}
\def\zh{\mbox{\boldmath$\hat{z}$}}
%
\def\Gammah{\mbox{$\hat{\Gamma}$}}
\def\Lambdah{\mbox{$\hat{\Lambda$}}}
\def\Sigmah{\mbox{$\hat{\Sigma$}}}
\def\Psih{\mbox{$\hat{\Psi$}}}
\def\Deltah{\mbox{$\hat{\Delta$}}}
\def\Xih{\mbox{$\hat{\Xi$}}}
\def\Upsilonh{\mbox{$\hat{\Upsilon$}}}
\def\Omegah{\mbox{$\hat{\Omega$}}}
\def\Thetah{\mbox{$\hat{\Theta$}}}
\def\Pih{\mbox{$\hat{\Pi$}}}
\def\Phih{\mbox{$\hat{\Phi$}}}
%
\def\bGammah{\mbox{\boldmath$\hat{\Gamma}$}}
\def\bLambdah{\mbox{\boldmath$\hat{\Lambda}$}}
\def\bSigmah{\mbox{\boldmath$\hat{\Sigma}$}}
\def\bPsih{\mbox{\boldmath$\hat{\Psi}$}}
\def\bDeltah{\mbox{\boldmath$\hat{\Delta}$}}
\def\bXih{\mbox{\boldmath$\hat{\Xi}$}}
\def\bUpsilonh{\mbox{\boldmath$\hat{\Upsilon}$}}
\def\bOmegah{\mbox{\boldmath$\hat{\Omega}$}}
\def\bThetah{\mbox{\boldmath$\hat{\Theta}$}}
\def\bPih{\mbox{\boldmath$\hat{\Pi}$}}
\def\bPhih{\mbox{\boldmath$\hat{\Phi}$}}
\def\Betah{\mbox{\boldmath$\hat{\eta}$}}
%
\def\define{\mbox{~$\stackrel{\triangle}{=}$~}}
\def\Del{\mbox{$\Delta$}}
\def\del{\mbox{$\delta$}}
\def\eps{\mbox{$\epsilon$}}
\def\QED{\hfill\mbox{Q.E.D}\par\endtrivlist\unskip}
%
\def\edot{\mbox{$\dot{\e}$}}
\def\eddot{\mbox{$\ddot{\e}$}}
\def\pdot{\mbox{$\dot{\p}$}}
\def\pddot{\mbox{$\ddot{\p}$}}
\def\qdot{\mbox{$\dot{\q}$}}
\def\qddot{\mbox{$\ddot{\q}$}}
\def\rdot{\mbox{$\dot{\r}$}}
\def\rddot{\mbox{$\ddot{\r}$}}
\def\xdot{\mbox{$\dot{\x}$}}
\def\xddot{\mbox{$\ddot{\x}$}}
\def\zdot{\mbox{$\dot{\z}$}}
\def\zddot{\mbox{$\ddot{\z}$}}
\def\Ddot{\mbox{$\dot{\D}$}}
\def\Hdot{\mbox{$\dot{\H}$}}
\def\Jdot{\mbox{$\dot{\J}$}}
\def\Ndot{\mbox{$\dot{\N}$}}
\def\Rdot{\mbox{$\dot{\R}$}}
\def\Tdot{\mbox{$\dot{\T}$}}
\def\Udot{\mbox{$\dot{\U}$}}
\def\Vdot{\mbox{$\dot{\V}$}}
\def\Wdot{\mbox{$\dot{\W}$}}
\def\Zdot{\mbox{$\dot{\Z}$}}
%%
%%   Reference Abbreviation
%%   ======================
%%
%\def\AR{{\em Advanced Robotics}}
%\def\ACC{{\em Proc. of American Control Conference}}
%\def\IJC{{\em Int. J. of Control}}
%\def\JMD{{\em J. of Mechanical Design}}
%\def\JRS{{\em J. of Robotic Systems}}
%\def\DSMC{{\em Trans. of the ASME J. of Dyn. Syst., Meas. and Contr.}}
%\def\MTAD{{\em Trans. of the ASME J. of Mechanisms, Transmission, and Automation
%    in Design}}
%\def\IJRR{{\em Int. J. of Robotics Research}}
%\def\IJRA{{\em IEEE J. of Robotics and Automation}}
%\def\ITRA{{\em IEEE Trans. on Robotics and Automation}}
%\def\ITAC{{\em IEEE Trans. on Automatic Control}}
%\def\ITIE{{\em IEEE Trans. on Industrial Electronics}}
%\def\ISMC{{\em IEEE Trans. Syst., Man, Cybern.}}
%\def\JIRS{{\em J. of Intelligent and Robotic System}}
%\def\JOTA{{\em J. of Opt. Theory and Applications}}
%\def\JRSJ{{\em J. of the Robotics Society of Japan}}
%\def\ICAR{{\em Proc. of Int. Conf. on Advanced Robotics}}
%\def\ICRA{{\em Proc. of IEEE Int. Conf. on Robotics and Automation}}
%\def\IROS{{\em Proc. of IEEE/RSJ Int. Conf. on Intelligent Robots and Systems}}
%\def\ISRR{{\em Int. Symp. of Robotic Research}}
%\def\PSMC{{\em Proc. of IEEE Int. Conf. on Syst., Man, Cybern.}}
%\def\JUSFA{{\em JAPAN--USA. Symp. on Flexible Automation}}
%\def\IECON{{\em Proc. of IEEE Int. Conf. on Industrial Electronics}}
%\def\CDC{{\em Proc. of Conf. on Decision and Control}}
%
%   New Command
%   ===========
%
\newcommand{\PD}[2]{\frac{\partial #1}{\partial #2}}

\newcommand{\EPSF}[2]{
    \begin{center}
    \begin{minipage}[t]{#1}     %   #1: Fig size
        \epsfxsize=#1\epsffile{#2}  %   #2: PS filename
    \end{minipage}
    \end{center}
}

\newcommand{\FBOXEPSF}[2]{
    \begin{minipage}[t]{#1}
        \epsfxsize=#1\epsffile{#2}      %       #2: PS filename
    \end{minipage}
}


%
%   Title page format for manuscript
%   ================================
%
%   Usage: \Cover{Title}{footnote}{authors}{pagestyle}{pagenumbering}
%
\newcommand{\Cover}[5]
{
    \thispagestyle{empty}
    \baselineskip 7.0ex
    \def\thefootnote{\fnsymbol{footnote}}
    \begin{center}
        {\Huge{\bf #1}}\footnote{#2}                \\
        \vspace*{3cm}
        \begin{Large}
            {\bf #3}                    \\
            \vspace*{3cm}
            {\bf J. Cheong}         \\
            \vspace*{3cm}
        \end{Large}
        \begin{Large}
            {\sf Robotics Lab.}             \\
            \vspace*{3ex}
            {\sf School of Mechanical Engineering}      \\
            \vspace*{3ex}
            {\sf Pohang University of Science \& Technology}
        \end{Large}
    \end{center}

    \newpage
    \setcounter{section}{0}
    \setcounter{page}{0}
    \setcounter{equation}{0}
    \pagestyle{#4}
    \pagenumbering{#5}
    \baselineskip 4.0ex         %   line spacing in text
    \jot    1.0em               %   row spacing in eqnarray
    \def\arraystretch{1.5}          %   row spacing in array
}
%
%   Title page format for summary
%   ================================
%
%   Usage: \NoteCover{Title}{Version}{pagestyle}{pagenumbering}
%
\newcommand{\NoteCover}[4]
{
    \Large
    \thispagestyle{empty}
    \baselineskip 2.5em
    \begin{center}
        {\Huge\bf #1}                   \\
	{\sf Version #2}		\\ 
      	\vspace*{5em}
        {\sl by Syungkwon Ra $\&$ Jaeyoung Han}               \\
        \vspace*{5em}
        {\sf \today}                   \\
        \vspace*{5em}
        {\sc Robotics Lab.}             \\
        \vspace*{1em}
        {\sc School of Mechanical $\&$ Aerospace Engineering}      \\
        \vspace*{1em}
        {\sc Seoul National University} 
    \end{center}

    \newpage
    \setcounter{section}{0}
    \setcounter{page}{0}
    \setcounter{equation}{0}
    \pagestyle{#3}
    \pagenumbering{#4}
    \normalsize
    \baselineskip 1.5em     %   line spacing in text
    \jot    4.0pt           %   row spacing in eqnarray
    \def\arraystretch{1.3}      %   row spacing in array
}

%
%   Title page format for summary
%   ================================
%
%   Usage: \Manuscript{heading}{Title}{pagestyle}{pagenumbering}
%
\newcommand{\Manuscript}[4]
{
    \thispagestyle{empty}
    \baselineskip 7.0ex
    \begin{flushleft}
        {\underline{\Large\sf #1}}
    \end{flushleft}
    \vspace*{1cm}
    \begin{center}
        {\Huge{\bf #2}}                     \\
        \vspace*{3cm}
        \begin{Large}
            {\bf by J. Cheong~ and~ J. Han}        \\
            \vspace*{3cm}
            \today                      \\
            \vspace*{3cm}
        \end{Large}
        \begin{Large}
            {\sf Robotics $\&$ Biomechatronics Lab.}             \\
            \vspace*{3ex}
            {\sf School of Mechanical Engineering}  \\
            \vspace*{3ex}
            {\sf Pohang University of Science \& Technology}
        \end{Large}
    \end{center}

    \newpage
    \setcounter{section}{0}
    \setcounter{page}{0}
    \setcounter{equation}{0}
    \pagestyle{#3}
    \pagenumbering{#4}
    \baselineskip 1.5em
    \jot    4.0pt           %   row spacing in eqnarray
    \def\arraystretch{1.5}      %   row spacing in array
}
%
%   Cover page format for Journal paper
%   ===================================
%
%   Usage: \JCover{where}{title}{pagestyle}
%
\newcommand{\JCover}[3]
{
    \pagestyle{empty}
    \setcounter{page}{0}
    \large
    \baselineskip 1.5em
    \begin{flushleft}
    \underline{\em #1}
    \end{flushleft}

    \vskip 5.0em

    \baselineskip 2.0em
    \begin{center}
    {\Large\bf #2}
    \end{center}

    \baselineskip 1.5em
    \vskip 2.0em

    \begin{center}
    Joono~Cheong$^{\ast}$,~
    Wan~Kyun~Chung$^{\dagger}$, ~and~
    Youngil~Youm$^{\dagger}$
  %  Youngil~Youm$^{\dagger}$, ~and~
  %  Wan~Kyun~Chung$^{\dagger}$
    \end{center}

    \vskip 4.0em

    \begin{quote}
    \begin{tabbing}
        $\ast$ \= ~~Postdoctoral Researcher, Mechanical Eng.,
        POSTECH:\\
    ~~~~~jncheong@postech.ac.kr \\
        $\dagger$ \> ~~Professor, School of Mech. Eng.,
            POSTECH:  \\
           ~~~~ wkchung@postech.ac.kr,~ youm@postech.ac.kr \\
   %        ~~~~ youm@postech.ac.kr,~ wkchung@postech.ac.kr \\
    \end{tabbing}
%   Please address all the correspondence to Professor~Youngil~Youm.
    \end{quote}
\vskip 1.0em
%~~The contents of this article are:
%\begin{itemize}
%\item Reply to the comments and questions by the referees.
%\item The revised manuscript.
%\end{itemize}
    \vskip 3.0em
%    \begin{quote}
%   Please address all the correspondence to Professor~Youngil~Youm
%~~~  Please address all the correspondence to Prof.~Wan~Kyun~Chung
%    \end{quote}
    \begin{quote}
%$\bullet$ Total number of figures : 9    \\
%$\bullet$ Total number of tables ~: 1     \\
%$\bullet$ Total number of pages   ~: 23
    \end{quote}
    \vskip 3.0em

    \begin{quote}
    Robotics \& Biomechatronics Lab.,\\Department of Mechanical Engineering,       \\
    Pohang University of Science \& Technology\,(POSTECH)       \\
    San--31,~Hyoja--Dong,~Pohang,~790--784,~\underline{\sc KOREA}   \\
    Fax:\,+82--54--279--5899,~~Tel:\,+82--54--279--2844
    \end{quote}

    \newpage
    \normalsize
    \setcounter{section}{0}
    \setcounter{page}{0}
    \setcounter{equation}{0}
    \pagestyle{#3}
    \pagenumbering{arabic}
    \baselineskip 1.5em
    \jot    4.0pt           %   row spacing in eqnarray
    \def\arraystretch{1.5}      %   row spacing in array
   % \def\thefootnote{\fnsymbol{footnote}}
}
%
%   Cover page format for Proceedings
%   =================================
%
%   Usage: \ProCover{where}{title}{pagestyle}
%
\newcommand{\ProCover}[3]
{
    \onecolumn
    \def\thefootnote{\fnsymbol{footnote}}
    \pagestyle{empty}
    \setcounter{page}{0}
    \large
    \baselineskip 1.5em
    \begin{flushleft}
    \underline{\sf #1}
    \end{flushleft}

    \vskip 5.0em

    \baselineskip 2.0em
    \begin{center}
    {\Large\bf #2}
    \end{center}

    \baselineskip 1.5em
    \vskip 2.0em

    \begin{center}
    Joono~Cheong$^{\ast}$,~
    Wan Kyun~Chung$^{\dagger}$,~ and~ Youngil~Youm$^{\ddagger}$~
    \end{center}
    \vskip 0.0em
{\small
    \begin{quote}
    \begin{tabbing}
        $\ast$ \= ~~Graduate student, School of Mechanical Engineering,
    POSTECH : jncheong@postech.ac.kr\\
        $\dagger$ \> ~~Professor, School of Mechanical
     Engineering, POSTECH : wkchung@postech.ac.kr                  \\
        $\ddagger$ \> ~~Professor, School of Mechanical Engineering,
    POSTECH : youm@postech.ac.kr \\
    \end{tabbing}
    \end{quote}

    \vskip 0.0em

%    \begin{quote}
%    Please address all the correspondence to Prof.~Wan Kyun~Chung
%    \end{quote}

    \vskip 0.0em

    \begin{quote}
    Robotics \& Bio-Mechatronics Lab.,~School of Mechanical Engineering,        \\
    Pohang University of Science \& Technology\,(POSTECH)       \\
    San--31,~Hyoja--Dong,~Pohang,~790--784,~\underline{\sc KOREA}   \\

    Fax:\,+82--54--279--5899,~~Tel:\,+82--54--279--2844
    \end{quote}
}
    \vskip 3.0em
    \setlength{\baselineskip}{0.5em}
\normalfont
\section*{\centering \bf Abstract}
We propose a PID composite controller for controlling
flexible arms modeled by singular perturbation approach. For slow
sub-controller, the PD plus disturbance observer is used, which
eventually takes on PID form. And for fast sub-controller, modal
feedback PID control is used. The integral action removes steady
state error from step disturbance and imperfect gravity
compensation though it complicates the analysis. We investigate
effects of design parameters in the controller to closed loop
response. Thereby, we can deliver a guideline for performance
tuning for general flexible systems. Through simulation and
experiments, the adequacy and performance of the proposed method
are verified.
%
\section*{\centering \bf Contribution}
The contribution of this paper has two folds; one is the
design of PID composite controller, and
the other is presenting an easy way for tuning performance
based on the designed controller.
To do this, for slow subsystem, disturbance observer is applied and
for fast subsystem, PID modal feedback is used.
We explain the roles of tuning parameters on the
overall performance. We present an illustrative
example by simulation and experiment.
This method is possibly utilized to multi-link flexible robots.

%%%%%%%%%%%%%%%%%%%%%%%%%%%%%%%%%%%%%%%%%%%%%%%%%%%%%%%%%%%%%%
%\section*{\centering \bf Abstract}
%%%  Put here the Abstract or Contribution   %
%In this paper, a simple direct parameter update rule is
%presented to suppress the vibration more quickly,
%considering the dynamics of flexible subsystem.
%Different from most of adaptive control schemes in multi-link
%flexible robots, which is
%indirect model-independent approach,
%the proposed adaptive control is a direct one and good for fast
%suppression of vibration of uncertain and untuned systems.
%Usually, the precise modeling of multi-link flexible robot is hard to
%obtain and even if we can get it, it is difficult to use in the on-line
%control tasks. Due to these reasons, a simplified model is used to
%describe the robot dynamics. The modelling errors and other structured
%uncertainty are considered as parameter perturbation.
%We verify the effectiveness of the proposed
%algorithm through experiments.
%%
%%
%\section*{\centering \bf Contribution}
%%%%%%%%%%%%%%%%%%%%%%%%%%%%%%%%%%%%%%%%%%%%%%%%%%%%%%%%%%%%%%%%%%%
%%%%%%%%%%% Below is the contribution of paper: ICRA2001  %%%%%%%%%
%%%%%%%%%%%%%%%%%%%%%%%%%%%%%%%%%%%%%%%%%%%%%%%%%%%%%%%%%%%%%%%%%%%
%We present a simple and efficient adaptive control law which improves the
%performance of vibration control algorithm.
%Many other adaptive approaches are based upon the signal processing
%technique using various filters and/or least square estimate.
%Thus, there is little physical meaning for adaptation.
%We formulate direct parameter update rule which minimizes the
%residual vibration using flexible motion equation.
%This was possible due to the application of bandwidth modulation
%technique to separate the whole system into
%rigid and flexible subsystems.
%By updating the parameters,
%the adaptive controller significantly and effectively reduces the
%residual vibration. Using this algorithm, the
%effort of time-consuming gain tuning can also be reduced because the
%parameters automatically change to accommodate the vibration.
%To robustify the
%algorithm, we combine the parameter projection and dead zone
%methods.  The experimental results showed the effectiveness of the
%proposed algorithm.
%%%%%%%%%%%%%%%%%%%%%%%%%%%%%%%%%%%%%%%%%%%%%%%%%%%%%%%%%%%%%%%%%%
%%%%%%%%%% Below is the contribution of paper: ISIM2000  %%%%%%%%%
%%%%%%%%%%%%%%%%%%%%%%%%%%%%%%%%%%%%%%%%%%%%%%%%%%%%%%%%%%%%%%%%%%
%Some dynamic properties of 3D flexible robot are obtained for the
%lumped mass spring model. The direct modal feedback controller is
%developed for the 3D vibration suppression for this model. To do
%this, we perform the instantaneous modal transformation to make
%the analysis and controller synthesis easier. After the modal
%transformation, the stable horizontal and vertical vibration
%suppression controller is obtained independently. We provides the
%stable symmetric feedback for the horizontal vibration control and
%direct inverse feedback for the vertical vibration control. Since
%during the anthropomorphic 3D motion, horizontal vibration control
%is not common and only a few works have reported the successful
%results, our proposed control can be a good solution in that kind
%of jobs.

%%%%%%%%%%%%%%%%%%%%%%%%%%%%%%%%%%%%%%%%%%%%%%%%%%%%%%%%%%%%%%%%%%
%%%%%%%%%% Below is the contribution of paper: ICRA2000  %%%%%%%%%
%%%%%%%%%%%%%%%%%%%%%%%%%%%%%%%%%%%%%%%%%%%%%%%%%%%%%%%%%%%%%%%%%%
%The closed loop bandwidth of rigid sub-system is defined using
%passivity-based approach and disturbance observer approach.
%We showed that there is a mutually exclusive
%relation between joint tracking performance and
%vibration suppression capability with single bandwidth parameter.
%We also propose a composite control strategy similar to singular
%perturbation approach for the isolated system.
%While the singular perturbation approach separates
%rigid/flexible subsystem by the different time scales,
%the proposed bandwidth modulation method separates two subsystems
%by using the operating frequency ranges not by time scale.
%Since the flexibly structured system shows the vibrational characteristics,
%its frequency is the most important information which should be considered.
%The proposed method fully utilizes this frequency information in the
%determination of the bandwidth of rigid subsystem, but the singular
%perturbation approach misses this information. Moreover, the proposed
%method provides effective design way when the frequency characteristics of
%external disturbances or unmodeled dynamics are known.
%We verify experimentally that the composite control for the
%isolated system is very good at vibration suppression.
%The proposed bandwidth modulation method is general and can be
%applied to any kind of flexible systems.
%
%
   % \newpage
    \normalsize
%    \twocolumn
%    \setcounter{section}{0}
%    \setcounter{page}{0}
%    \setcounter{equation}{0}
%    \pagestyle{#3}
%    \pagenumbering{arabic}
%%   \baselineskip 1.35em
%    \baselineskip 1.3em
%    \jot    4.0pt           %   row spacing in eqnarray
%    \def\arraystretch{1.5}      %   row spacing in array
}

%
%   New Environment
%   ===============
%
%   Usage: \begin{Assume}{(A.\arabic{newctr})}{newctr}
%
\newenvironment{Assume}[2]{\newcounter{#2}\begin{list}{#1}
{\usecounter{#2}\setlength{\rightmargin}{\leftmargin}}}{\end{list}}

\setlength\arraycolsep{1.2pt}       %   eqnarray column separation
\renewcommand\figurename{Fig.}
\def\fps@figure{h}
